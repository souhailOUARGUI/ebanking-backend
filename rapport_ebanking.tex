\documentclass[12pt,a4paper]{report}

% Packages essentiels
\usepackage[utf8]{inputenc}
\usepackage[T1]{fontenc}
\usepackage[french]{babel}
\usepackage{graphicx}
\usepackage{hyperref}
\usepackage{listings}
\usepackage{xcolor}
\usepackage{float}
\usepackage{geometry}
\usepackage{titlesec}
\usepackage{amsmath}
\usepackage{fancyhdr}
\usepackage{pdfpages}

% Configuration de la géométrie de la page
\geometry{
    a4paper,
    top=2.5cm,
    bottom=2.5cm,
    left=2.5cm,
    right=2.5cm
}

% Configuration des listings pour le code source
\definecolor{codegreen}{rgb}{0,0.6,0}
\definecolor{codegray}{rgb}{0.5,0.5,0.5}
\definecolor{codepurple}{rgb}{0.58,0,0.82}
\definecolor{backcolour}{rgb}{0.95,0.95,0.92}

\lstdefinestyle{mystyle}{
    backgroundcolor=\color{backcolour},
    commentstyle=\color{codegreen},
    keywordstyle=\color{magenta},
    numberstyle=\tiny\color{codegray},
    stringstyle=\color{codepurple},
    basicstyle=\ttfamily\small,
    breakatwhitespace=false,
    breaklines=true,
    captionpos=b,
    keepspaces=true,
    numbers=left,
    numbersep=5pt,
    showspaces=false,
    showstringspaces=false,
    showtabs=false,
    tabsize=2
}

\lstset{style=mystyle}

% Configuration des en-têtes et pieds de page
% \pagestyle{fancy}
% \fancyhf{}
% \fancyhead[L]{ENSET Mohammedia}
% \fancyhead[R]{Projet Architecture JEE}
% \fancyfoot[C]{\thepage}

% % Titre du document
% \title{
%     \vspace{-2.5cm}
%     \begin{center}
%         % Logo placé en haut de la page
%         \includegraphics[width=0.3\textwidth]{image1.png}\\[2cm]
%         \textbf{\Large Rapport de Projet\\Architecture JEE et Systèmes Distribués}\\
%         \vspace{0.5cm}
%         \large Application de Gestion Bancaire (E-Banking)
%     \end{center}
% }
% \author{
%     \vspace{1cm}
%     \textbf{Encadré par: Pr. Mohamed YOUSSFI}\\[0.3cm]
%     \textbf{Réalisé par: [Votre Nom]}\\[0.3cm]
%     \textit{GLSID}\\[0.3cm]
%     \textit{ENSET Mohammedia, Université Hassan II}
% }
% \date{\today}






\begin{document}


\includepdf[pages={1}]{page_de_garde.pdf}




\tableofcontents
\thispagestyle{empty}
\newpage

\chapter*{Introduction}
\addcontentsline{toc}{chapter}{Introduction}
\section{Contexte du projet}
Ce projet consiste en la réalisation d'une application de gestion bancaire (E-Banking) permettant de gérer des clients, des comptes bancaires (courants et épargnes) et des opérations bancaires (retraits, versements, virements). L'application offre une interface REST pour interagir avec ces différentes entités.

\section{Technologies Utilisées}
Voici les technologies et frameworks intégrés dans le projet:
\begin{itemize}
    \item Spring Boot 3.4.5
    \item Spring Data JPA
    \item Spring MVC avec REST
    \item MySQL Database
    \item Lombok
    \item Swagger/OpenAPI pour la documentation
    \item Angular (pour la partie frontend)
    \item JWT pour l'authentification
\end{itemize}

\chapter{Architecture de l'application}
\section{Architecture technique}
L'application E-Banking suit une architecture en couches classique d'une application Spring Boot:

\begin{figure}[H]
    \centering
    % Remplacez par votre image d'architecture
    \includegraphics[width=0.8\textwidth]{architecture.png}
    \caption{Architecture technique de l'application E-Banking}
    \label{fig:architecture}
\end{figure}

L'architecture se compose des couches suivantes:
\begin{itemize}
    \item \textbf{Couche Présentation}: Contrôleurs REST exposant les API
    \item \textbf{Couche Service}: Logique métier et orchestration des opérations
    \item \textbf{Couche DAO}: Accès aux données via les repositories Spring Data JPA
    \item \textbf{Couche Entités}: Modèles de données JPA
    \item \textbf{Couche DTO}: Objets de transfert de données pour les API
    \item \textbf{Couche Mappers}: Conversion entre entités et DTOs
\end{itemize}

\section{Modèle de données}
Le diagramme de classe ci-dessous représente les entités principales de l'application:

\begin{figure}[H]
    \centering
    % Remplacez par votre diagramme de classe
    \includegraphics[width=0.8\textwidth]{ebankingClassDia.png}
    \caption{Diagramme de classe de l'application E-Banking}
    \label{fig:class-diagram}
\end{figure}

Les principales entités sont:
\begin{itemize}
    \item \textbf{Customer}: Représente un client de la banque
    \item \textbf{BankAccount}: Classe abstraite représentant un compte bancaire
    \item \textbf{CurrentAccount}: Compte courant avec possibilité de découvert
    \item \textbf{SavingAccount}: Compte épargne avec taux d'intérêt
    \item \textbf{AccountOperation}: Opérations effectuées sur un compte
\end{itemize}

\chapter{Implémentation de la couche DAO}
\section{Entités JPA}
Voici les principales entités JPA implémentées:

\subsection{Entité Customer}
\begin{lstlisting}[language=Java, caption=Entité Customer.java]
@Entity
@Data @NoArgsConstructor @AllArgsConstructor
public class Customer {
    @Id
    @GeneratedValue(strategy = GenerationType.IDENTITY)
    private Long id;
    private String name;
    private String email;
    @OneToMany(mappedBy = "customer")
    private List<BankAccount> bankAccountsList;
}
\end{lstlisting}

\subsection{Entité BankAccount}
\begin{lstlisting}[language=Java, caption=Entité BankAccount.java]
@Entity
@Inheritance(strategy = InheritanceType.SINGLE_TABLE)
@DiscriminatorColumn(name = "TYPE", length = 4)
@Data @NoArgsConstructor @AllArgsConstructor
public class BankAccount {
    @Id
    private String id;
    private Date createdAt;
    private double balance;
    private String currency;
    @Enumerated(EnumType.STRING)
    private AccountStatus status;
    @ManyToOne
    private Customer customer;
    @OneToMany(mappedBy = "bankAccount", fetch = FetchType.LAZY)
    private List<AccountOperation> accountOperationsList;
}
\end{lstlisting}

\subsection{Entité CurrentAccount}
\begin{lstlisting}[language=Java, caption=Entité CurrentAccount.java]
@Entity
@DiscriminatorValue("CA")
@Data @AllArgsConstructor @NoArgsConstructor
public class CurrentAccount extends BankAccount{
    private double overdraft;
}
\end{lstlisting}

\subsection{Entité SavingAccount}
\begin{lstlisting}[language=Java, caption=Entité SavingAccount.java]
@Entity
@DiscriminatorValue("SA")
@Data @AllArgsConstructor @NoArgsConstructor
public class SavingAccount extends BankAccount{
    private double interestRate;
}
\end{lstlisting}

\subsection{Entité AccountOperation}
\begin{lstlisting}[language=Java, caption=Entité AccountOperation.java]
@Entity
@Data @NoArgsConstructor @AllArgsConstructor
public class AccountOperation {
    @Id @GeneratedValue(strategy = GenerationType.IDENTITY)
    private Long id;
    private Date operationDate;
    private double amount;
    private String description;
    @Enumerated(EnumType.STRING)
    private OperationType operationType;
    @ManyToOne
    private BankAccount bankAccount;
}
\end{lstlisting}

\section{Interfaces Repository}
Voici les interfaces Repository pour l'accès aux données:

\begin{lstlisting}[language=Java, caption=CustomerRepository.java]
public interface CustomerRepository extends JpaRepository<Customer, Long> {
}
\end{lstlisting}

\begin{lstlisting}[language=Java, caption=BankAccountRepository.java]
public interface BankAccountRepository extends JpaRepository<BankAccount, String> {
}
\end{lstlisting}

\begin{lstlisting}[language=Java, caption=AccountOperationRepository.java]
public interface AccountOperationRepository extends JpaRepository<AccountOperation, Long> {
    List<AccountOperation> findByBankAccountId(String accountId);
    Page<AccountOperation> findByBankAccountId(String accountId, Pageable pageable);
}
\end{lstlisting}

\section{Initialisation des données}
L'application utilise un CommandLineRunner pour initialiser des données de test:

\begin{lstlisting}[language=Java, caption=Initialisation des données]
@Bean
CommandLineRunner commandLineRunner(BankAccountService bankAccountService){
    return args -> {
        // Création de clients
        Stream.of("amine","aziz","Mohamed").forEach(name->{
            CustomerDTO customer=new CustomerDTO();
            customer.setName(name);
            customer.setEmail(name+"@gmail.com");
            bankAccountService.saveCustomer(customer);
        });

        // Création de comptes bancaires
        bankAccountService.listCustomers().forEach(customer->{
            try {
                bankAccountService.saveCurrentBankAccount(Math.random()*90000,9000,customer.getId());
                bankAccountService.saveSavingBankAccount(Math.random()*120000,5.5,customer.getId());
            } catch (CustomerNotFoundException e) {
                e.printStackTrace();
            }
        });

        // Création d'opérations
        List<BankAccountDTO> bankAccounts = bankAccountService.bankAccountList();
        for (BankAccountDTO bankAccount:bankAccounts){
            for (int i = 0; i <10 ; i++) {
                String accountId;
                if(bankAccount instanceof SavingBankAccountDTO){
                    accountId=((SavingBankAccountDTO) bankAccount).getId();
                } else{
                    accountId=((CurrentBankAccountDTO) bankAccount).getId();
                }
                bankAccountService.credit(accountId,10000+Math.random()*120000,"Credit");
                bankAccountService.debit(accountId,1000+Math.random()*9000,"Debit");
            }
        }
    };
}
\end{lstlisting}

\chapter{Implémentation de la couche Service}
\section{Interface BankAccountService}
L'interface définit les opérations métier disponibles:

\begin{lstlisting}[language=Java, caption=BankAccountService.java]
public interface BankAccountService {
    CustomerDTO saveCustomer(CustomerDTO customerDTO);
    CurrentBankAccountDTO saveCurrentBankAccount(double initialBalance, double overDraft, Long customerId) throws CustomerNotFoundException;
    SavingBankAccountDTO saveSavingBankAccount(double initialBalance, double interestRate, Long customerId) throws CustomerNotFoundException;
    List<CustomerDTO> listCustomers();
    BankAccountDTO getBankAccount(String accountId) throws BankAccountNotFoundException;
    void debit(String accountId, double amount, String description) throws BankAccountNotFoundException, BalanceNotSufficientException;
    void credit(String accountId, double amount, String description) throws BankAccountNotFoundException;
    void transfer(String accountIdSource, String accountIdDestination, double amount) throws BankAccountNotFoundException, BalanceNotSufficientException;
    List<BankAccountDTO> bankAccountList();
    CustomerDTO getCustomer(Long customerId) throws CustomerNotFoundException;
    CustomerDTO updateCustomer(CustomerDTO customerDTO);
    void deleteCustomer(Long customerId);
    List<AccountOperationDTO> accountHistory(String accountId);
    AccountHistoryDTO getAccountHistory(String accountId, int page, int size) throws BankAccountNotFoundException;
}
\end{lstlisting}

\section{Implémentation BankAccountServiceImpl}
Voici quelques méthodes clés de l'implémentation du service:

\begin{lstlisting}[language=Java, caption=BankAccountServiceImpl.java (extrait)]
@Service
@Transactional
@AllArgsConstructor
@Slf4j
public class BankAccountServiceImpl implements BankAccountService {
    private BankAccountRepository bankAccountRepository;
    private CustomerRepository customerRepository;
    private AccountOperationRepository accountOperationRepository;
    private BankAccountMapperImpl dtoMapper;

    @Override
    public CustomerDTO saveCustomer(CustomerDTO customerDTO) {
        log.info("Saving new Customer");
        Customer customer = dtoMapper.fromCustomerDTO(customerDTO);
        Customer savedCustomer = customerRepository.save(customer);
        return dtoMapper.fromCustomer(savedCustomer);
    }

    @Override
    public CurrentBankAccountDTO saveCurrentBankAccount(double initialBalance, double overDraft, Long customerId) throws CustomerNotFoundException {
        Customer customer = customerRepository.findById(customerId).orElse(null);
        if (customer == null) {
            throw new CustomerNotFoundException("Customer not found");
        }
        CurrentAccount currentAccount = new CurrentAccount();
        currentAccount.setId(UUID.randomUUID().toString());
        currentAccount.setCreatedAt(new Date());
        currentAccount.setBalance(initialBalance);
        currentAccount.setCurrency("USD");
        currentAccount.setOverdraft(overDraft);
        currentAccount.setCustomer(customer);
        CurrentAccount savedAccount = bankAccountRepository.save(currentAccount);
        return dtoMapper.fromCurrentBankAccount(savedAccount);
    }

    @Override
    public void credit(String accountId, double amount, String description) throws BankAccountNotFoundException {
        BankAccount bankAccount = bankAccountRepository.findById(accountId)
            .orElseThrow(() -> new BankAccountNotFoundException("Bank Account not found"));
        AccountOperation accountOperation = new AccountOperation();
        accountOperation.setOperationType(OperationType.CREDIT);
        accountOperation.setAmount(amount);
        accountOperation.setOperationDate(new Date());
        accountOperation.setDescription(description);
        accountOperation.setBankAccount(bankAccount);
        accountOperationRepository.save(accountOperation);
        bankAccount.setBalance(bankAccount.getBalance() + amount);
        bankAccountRepository.save(bankAccount);
    }

    // Autres méthodes...
}
\end{lstlisting}

\chapter{Implémentation de la couche Web}
\section{Contrôleurs REST}

\subsection{CustomerRestController}
\begin{lstlisting}[language=Java, caption=CustomerRestController.java]
@RestController
@AllArgsConstructor
@Slf4j
public class CustomerRestController {
    private BankAccountService bankAccountService;

    @GetMapping("/customers")
    public List<CustomerDTO> customers() {
        return bankAccountService.listCustomers();
    }

    @GetMapping("/customers/{id}")
    public CustomerDTO getCustomer(@PathVariable(name = "id") Long customerId) throws CustomerNotFoundException {
        return bankAccountService.getCustomer(customerId);
    }

    @PostMapping("/customers")
    public CustomerDTO saveCustomer(@RequestBody CustomerDTO customerDTO) {
        return bankAccountService.saveCustomer(customerDTO);
    }

    @PutMapping("/customers/{customerId}")
    public CustomerDTO updateCustomer(@PathVariable Long customerId, @RequestBody CustomerDTO customerDTO) {
        customerDTO.setId(customerId);
        return bankAccountService.updateCustomer(customerDTO);
    }

    @DeleteMapping("/customers/{id}")
    public void deleteCustomer(@PathVariable Long id) {
        bankAccountService.deleteCustomer(id);
    }
}
\end{lstlisting}

\subsection{BankAccountRestApi}
\begin{lstlisting}[language=Java, caption=BankAccountRestApi.java]
@RestController
@AllArgsConstructor
public class BankAccountRestApi {
    private BankAccountService bankAccountService;

    @GetMapping("/accounts/{accountId}")
    public BankAccountDTO getBankAccount(@PathVariable String accountId) throws BankAccountNotFoundException {
        return bankAccountService.getBankAccount(accountId);
    }

    @GetMapping("/accounts")
    public List<BankAccountDTO> listAccounts() {
        return bankAccountService.bankAccountList();
    }

    @GetMapping("/accounts/{accountId}/operations")
    public List<AccountOperationDTO> getHistory(@PathVariable String accountId) {
        return bankAccountService.accountHistory(accountId);
    }

    @GetMapping("/accounts/{accountId}/pageOperations")
    public AccountHistoryDTO getAccountHistory(
            @PathVariable String accountId,
            @RequestParam(name="page", defaultValue="0") int page,
            @RequestParam(name="size", defaultValue="5") int size) throws BankAccountNotFoundException {
        return bankAccountService.getAccountHistory(accountId, page, size);
    }
}
\end{lstlisting}

\section{Configuration CORS}
Pour permettre les requêtes cross-origin depuis le frontend Angular:

\begin{lstlisting}[language=Java, caption=Configuration CORS]
@Bean
public CorsFilter corsFilter() {
    CorsConfiguration config = new CorsConfiguration();
    config.setAllowCredentials(true);
    config.addAllowedOrigin("http://localhost:4200");
    config.setAllowedHeaders(Arrays.asList("Origin", "Access-Control-Allow-Origin", "Content-Type",
            "Accept", "Authorization", "Origin,Accept", "X-Request-With", "Access-Control-Request-Method",
            "Access-Control-Request-Headers"));
    config.setExposedHeaders(Arrays.asList("Origin", "Content-Type", "Accept", "Authorization", "Access-Control-Allow-Origin",
            "Access-Control-Allow-Origin", "Access-Control-Allow-Credentials"));
    config.setAllowedMethods(Arrays.asList("GET", "POST", "PUT", "DELETE", "OPTIONS"));

    UrlBasedCorsConfigurationSource urlBasedCorsConfigurationSource = new UrlBasedCorsConfigurationSource();
    urlBasedCorsConfigurationSource.registerCorsConfiguration("/**", config);
    return new CorsFilter(urlBasedCorsConfigurationSource);
}
\end{lstlisting}

\chapter{Gestion des DTOs et Mappers}
\section{Data Transfer Objects (DTOs)}
Les DTOs permettent de transférer les données entre les couches sans exposer les entités JPA:

\begin{lstlisting}[language=Java, caption=CustomerDTO.java]
@Data
public class CustomerDTO {
    private Long id;
    private String name;
    private String email;
}
\end{lstlisting}

\begin{lstlisting}[language=Java, caption=BankAccountDTO.java]
@Data
public abstract class BankAccountDTO {
    private String type;
}
\end{lstlisting}

\begin{lstlisting}[language=Java, caption=CurrentBankAccountDTO.java]
@Data
public class CurrentBankAccountDTO extends BankAccountDTO {
    private String id;
    private String currency;
    private Date createdAt;
    private AccountStatus status;
    private CustomerDTO customer;
    private double overdraft;
}
\end{lstlisting}

\begin{lstlisting}[language=Java, caption=SavingBankAccountDTO.java]
@Data
public class SavingBankAccountDTO extends BankAccountDTO {
    private String id;
    private String currency;
    private Date createdAt;
    private AccountStatus status;
    private CustomerDTO customer;
    private double interestRate;
}
\end{lstlisting}

\section{Mappers}
Les mappers assurent la conversion entre entités et DTOs:

\begin{lstlisting}[language=Java, caption=BankAccountMapperImpl.java]
@Service
public class BankAccountMapperImpl {
    public CustomerDTO fromCustomer(Customer customer) {
        CustomerDTO customerDTO = new CustomerDTO();
        BeanUtils.copyProperties(customer, customerDTO);
        return customerDTO;
    }

    public Customer fromCustomerDTO(CustomerDTO customerDTO) {
        Customer customer = new Customer();
        BeanUtils.copyProperties(customerDTO, customer);
        return customer;
    }

    public SavingBankAccountDTO fromSavingBankAccount(SavingAccount savingAccount) {
        SavingBankAccountDTO savingBankAccountDTO = new SavingBankAccountDTO();
        BeanUtils.copyProperties(savingAccount, savingBankAccountDTO);
        savingBankAccountDTO.setCustomer(fromCustomer(savingAccount.getCustomer()));
        savingBankAccountDTO.setType(savingAccount.getClass().getSimpleName());
        return savingBankAccountDTO;
    }

    public CurrentBankAccountDTO fromCurrentBankAccount(CurrentAccount currentAccount) {
        CurrentBankAccountDTO currentBankAccountDTO = new CurrentBankAccountDTO();
        BeanUtils.copyProperties(currentAccount, currentBankAccountDTO);
        currentBankAccountDTO.setCustomer(fromCustomer(currentAccount.getCustomer()));
        currentBankAccountDTO.setType(currentAccount.getClass().getSimpleName());
        return currentBankAccountDTO;
    }

    // Autres méthodes de mapping...
}
\end{lstlisting}

\chapter{Gestion des exceptions}
\section{Exceptions personnalisées}
L'application définit des exceptions métier personnalisées:

\begin{lstlisting}[language=Java, caption=CustomerNotFoundException.java]
public class CustomerNotFoundException extends Exception {
    public CustomerNotFoundException(String message) {
        super(message);
    }
}
\end{lstlisting}

\begin{lstlisting}[language=Java, caption=BankAccountNotFoundException.java]
public class BankAccountNotFoundException extends Exception {
    public BankAccountNotFoundException(String message) {
        super(message);
    }
}
\end{lstlisting}

\begin{lstlisting}[language=Java, caption=BalanceNotSufficientException.java]
public class BalanceNotSufficientException extends Exception {
    public BalanceNotSufficientException(String message) {
        super(message);
    }
}
\end{lstlisting}

\section{Gestionnaire global d'exceptions}
Un gestionnaire global d'exceptions pourrait être implémenté pour traiter les erreurs de manière uniforme.

\chapter{Configuration et déploiement}
\section{Configuration de la base de données}
La configuration de la base de données est définie dans le fichier application.properties:

\begin{lstlisting}[language=properties, caption=application.properties]
spring.application.name=ebanking-backend
server.port=8085
spring.datasource.url=jdbc:mysql://localhost:3306/E-BANK?createDatabaseIfNotExist=true
spring.datasource.username=root
spring.datasource.password=
spring.jpa.hibernate.ddl-auto=update
spring.jpa.properties.hibernate.dialect=org.hibernate.dialect.MariaDBDialect
spring.jpa.show-sql=true
\end{lstlisting}

\section{Dépendances Maven}
Les principales dépendances du projet sont définies dans le fichier pom.xml:

\begin{lstlisting}[language=XML, caption=pom.xml (extrait)]
<dependencies>
    <dependency>
        <groupId>org.springframework.boot</groupId>
        <artifactId>spring-boot-starter-data-jpa</artifactId>
    </dependency>
    <dependency>
        <groupId>org.springframework.boot</groupId>
        <artifactId>spring-boot-starter-web</artifactId>
    </dependency>
    <dependency>
        <groupId>org.springframework.boot</groupId>
        <artifactId>spring-boot-devtools</artifactId>
        <scope>runtime</scope>
        <optional>true</optional>
    </dependency>
    <dependency>
        <groupId>com.h2database</groupId>
        <artifactId>h2</artifactId>
        <scope>runtime</scope>
    </dependency>
    <dependency>
        <groupId>com.mysql</groupId>
        <artifactId>mysql-connector-j</artifactId>
        <scope>runtime</scope>
    </dependency>
    <dependency>
        <groupId>org.springdoc</groupId>
        <artifactId>springdoc-openapi-starter-webmvc-ui</artifactId>
        <version>2.8.8</version>
    </dependency>
    <dependency>
        <groupId>org.projectlombok</groupId>
        <artifactId>lombok</artifactId>
        <optional>true</optional>
    </dependency>
</dependencies>
\end{lstlisting}

% \chapter{Tests et validation}
% \section{Tests unitaires}
% Des tests unitaires pourraient être implémentés pour valider le comportement des services et des contrôleurs.

% \section{Tests d'intégration}
% Des tests d'intégration pourraient être réalisés pour valider le fonctionnement de l'application dans son ensemble.

\chapter{Documentation de l'API}
\section{Swagger/OpenAPI}
L'API est documentée à l'aide de Swagger/OpenAPI, accessible à l'URL: http://localhost:8085/swagger-ui.html


\begin{figure}[H]
    \centering
    % Remplacez par votre diagramme de classe
    \includegraphics[width=0.8\textwidth]{swagger_doc.png}
    \caption{Swagger API documentation}
    \label{fig:Swagger_API_documentation}
\end{figure}

\section{Test de GET method}
\begin{figure}[H]
    \centering
    % Remplacez par votre diagramme de classe
    \includegraphics[width=0.8\textwidth]{getCust.png}
    \caption{Get request}
    \label{fig:Get_request}
\end{figure}


\section{Test de POST method}

\begin{figure}[H]
    \centering
    % Remplacez par votre diagramme de classe
    \includegraphics[width=0.8\textwidth]{postCust.png}
    \caption{POST request}
    \label{fig:Get_request}
\end{figure}

Et voila le resultat de la requete:

\begin{figure}[H]
    \centering
    % Remplacez par votre diagramme de classe
    \includegraphics[width=0.8\textwidth]{postCustREsult.png}
    \caption{Resultat de Get request}
    \label{fig:Resultat_de_Get_request}
\end{figure}



\chapter*{Conclusion}
\addcontentsline{toc}{chapter}{Conclusion}
\section{Bilan}
Ce projet a permis de mettre en œuvre une application bancaire complète en utilisant les technologies Spring Boot et Angular. L'architecture en couches adoptée facilite la maintenance et l'évolution de l'application. Les fonctionnalités principales de gestion des clients, des comptes et des opérations bancaires ont été implémentées avec succès.

\section{Perspectives d'amélioration}
Plusieurs améliorations pourraient être apportées à l'application:
\begin{itemize}
    \item Implémentation d'un système d'authentification et d'autorisation avec JWT
    \item Ajout de fonctionnalités de reporting et de statistiques
    \item Mise en place de tests automatisés plus complets
    \item Amélioration de l'interface utilisateur
    \item Déploiement dans un environnement de production avec Docker et Kubernetes
\end{itemize}

\begin{thebibliography}{9}
\bibitem{spring} Spring Framework Documentation, \url{https://spring.io/docs}
\bibitem{hibernate} Hibernate Documentation, \url{https://hibernate.org/orm/documentation/}
\bibitem{angular} Angular Documentation, \url{https://angular.io/docs}
\end{thebibliography}

\end{document}